%----以下文字の大きさ,フォント,余白などを決定する設定----
\documentclass[a4paper,10pt]{jsarticle}
\bibliographystyle{junsrt} %参考文献の並べ順junsrtの場合,本文内で参照した順
\setlength{\topmargin}{-17truemm}
\setlength{\oddsidemargin}{-0.4truemm}
\setlength{\textwidth}{160truemm}
\setlength{\textheight}{247truemm}
%------usepackage-----------------------------------
\usepackage{listings,jlisting}%ソースコードを埋め込むためのパッケージ(jlistingはサイトからDLする必要あり)
\usepackage[dvipdfmx]{hyperref}%しおり作成,URL作成,ハイパーリンク機能のためのパッケージ ※今回のチャプターに直接関係はありません
\usepackage{pxjahyper}
%-------------------------------------------------
\lstset{%ソースコードを書き込む時の設定--------------------------------
  columns=[l]{fullflexible}, % 文字をつめる
  basicstyle=\footnotesize,%
  commentstyle=\textit,%
  classoffset=1,%
  keywordstyle=\bfseries,%
  frame=single,framesep=5pt,%枠線の種類 frame=singleならば一重枠,frame=tRBlならば二重枠
  showstringspaces=false,%
  numbers=left,stepnumber=1,numberstyle=\footnotesize,%
  breaklines=true,%trueで1行が長い時に改行
  firstnumber=1,%行番号が何番から始まるか
  xrightmargin=1zw,%枠右側の余白 
  xleftmargin=1zw%枠左側の余白
}%本文中に \lstinputlisting[caption=キャプション,label=ラベル]{ファイル名}でソースコードを貼り付け
%--------------------------------------------------------------------

\title{以降のchapterについて}%タイトル
\author{ER18028 久保 宏樹}%著者
\date{2020年12月20日}%日付

\begin{document}%本文の始まり
\maketitle
\section{注意点}
以降のchapterでは「semi/homework/2020\_12\_14/2020\_12\_14.ipynb」で環境構築をした前提で説明を行います.
また,texファイルはVScodeを使用して記述する前提とします.\\

一部のpdf閲覧ツール(Adobe Acrobat Reader DC等)の場合,pdfを開いたままの状態でコンパイルするとコンパイルエラーが発生するので注意してください.
SumatraPDFというpdf閲覧ツールの場合,pdfを開いたままの状態でもコンパイルエラーが出ないのでおすすめです.\\

各chapterには必須の課題と,任意の課題があります.\\
必須の課題は全て「著者の名前を自分の名前に変更する,日付を提出日に変更する」になっています.
これは,各chapterのtexファイルが正しくコンパイルできるかを確認するためです.\\
任意の課題は,各chapterに沿った内容の課題になっています.4年生になると必ず使う内容のものがほとんどなので,余裕があれば挑戦しましょう.

\section{提出期限について}
提出の際は,各chapterのtexファイル,コンパイルした後のpdfファイルを提出してください
(その他コンパイルの際に生成されるファイルはあってもなくてもOKです).
また,ディレクトリ名を「tex\_lec」→「er18○○○tex\_lec」に変更して提出してください.\\
githubへのアップロードの方法はゼミの時間に説明してあるので,忘れてしまった人はその資料を参考にしましょう.\\
chapter1~chapter7までの課題の提出期限は12月20日(日)の18時までとします.
今回の課題は「texの環境が正しく構築されていることを確認する」ことが最優先なので,各chapterの必須の課題を優先的に行ってください.
\end{document}