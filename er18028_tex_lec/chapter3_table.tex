%----以下文字の大きさ,フォント,余白などを決定する設定----
\documentclass[a4paper,10pt]{jsarticle}
\setlength{\topmargin}{-17truemm}
\setlength{\oddsidemargin}{-0.4truemm}
\setlength{\textwidth}{160truemm}
\setlength{\textheight}{247truemm}
%-------------------------------------------------

%\usepackage{}:Latexで拡張機能を使用するための宣言-------------
\usepackage[dvipdfmx]{graphicx}%図を埋め込むためのパッケージ
\usepackage{float}%その場に表・図を表示させるパッケージ
%------------------------------------------------------------------

\title{表を表示する}%タイトル
\author{ER18028 久保 宏樹}%著者
\date{2020年12月20日}%日付

\begin{document}%本文の始まり
\maketitle%上記で設定したtitle,author,dateを表示
\section{表の表示について}
chapter3\_table.texとchapter3\_table.pdfを照らし合わせて読んでください.\\

usepackageでdvipdfmx, graphicx, floatを指定すると表を作成することが可能\\

表は,表\ref{tb:label}のように参照することが可能です.基本的に,表を記述する場合,必ず1度以上参照する必要があります.
また,一度コンパイルしただけだと,参照した時に「?」と表示されるので「pLaTeX(ptex2pdf)」→「pLaTeX(ptex2pdf)」と2回コンパイルを実行する必要があります.\\
※ VScodeでの環境構築が完了している場合,「ctrl+s」で保存するだけでコンパイルが完了します.
%---------------表の表示方法--------------------
\begin{table}[H]
    \caption{title}
    \label{tb:label}                 
    \centering                     
    \begin{tabular}{c | c  c}
        \hline                        
        title1 & title2 & title2-2 \\          
        \hline \hline
        hoge1 & hoge2 & hoge2-2 \\
        hoge3 & hoge4 & hoge4-2 \\
        \hline
    \end{tabular}
\end{table}

\begin{table}[H]
    \caption{title}
    \label{tb:label}                 
    \centering                     
    \begin{tabular}{c | c  c}
        \hline                        
        title1 & title2 & title2-2 \\          
        \hline \hline
        hoge1 & hoge2 & hoge2-2 \\
        hoge3 & hoge4 & hoge4-2 \\
        \hline
    \end{tabular}
\end{table}
%------------------------------------------
注意点:左端,右端に縦線を加えることは基本的に禁止です.

%----------------以下課題------------------------------------
\section{課題}
\noindent 必須:authorを自分の名前,dateを提出日に変更してください.\\
任意:表を追加してください(表は何でも可).ただし,左端,右端に縦線を加えることは禁止とします.また,refを使用して「表を表\ref{tb:label}に示す」の様に参照してください.\\

\noindent 編集が完了したら,コンパイルしてください.\\
VScodeで編集している場合,「ctrl+s」で保存&コンパイル→「ctrl+alt+V」で出力されるpdfをプレビューできるので変更点が反映されていることを確認してください.
%-------------------------------------------------------------
\end{document}

%以下,今回登場した機能の説明
%------------------------表-------------------------------------
%\begin{table}[表の表示位置]     % 表の表示位置の例→h:その場,b:ページ下部,t:ページ上部,p:他のページ
%   \centering                   % centering:中心に表示
%   \caption{表のタイトル}
%   \label{ラベル名}       %表の場合は\label{tb:ラベル名}で定義することが多い.ラベルは\ref{ラベル名}で参照可能.
%   \begin{tabular}{各列の設定} % begin{tabular}{|c|c|}のようにバーティカルバーを挟むと縦線が表示される.
%       \centering        % 各列の表示位置の例→l:左詰,c:中央揃え,r:右揃え
%       \hline                  % hline:横線を追加
%       title1 & title2 \\      % 1列目 & 2 列目 \\ 
%       \hline \hline
%       hoge1 & hoge2 \\
%       hoge3 & hoge4 \\
%    \end{tabular}
%\end{table}
%------------------------------------------------------------