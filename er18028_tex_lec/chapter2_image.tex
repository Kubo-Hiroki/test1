%----以下文字の大きさ,フォント,余白などを決定する設定----
\documentclass[a4paper,10pt]{jsarticle}
\setlength{\topmargin}{-17truemm}
\setlength{\oddsidemargin}{-0.4truemm}
\setlength{\textwidth}{160truemm}
\setlength{\textheight}{247truemm}
%-------------------------------------------------

%\usepackage{}:Latexで拡張機能を使用するための宣言-------------
\usepackage[dvipdfmx]{graphicx}%図を埋め込むためのパッケージ
\usepackage{float}%その場に表・図を表示させるパッケージ
%------------------------------------------------------------------

\title{画像を表示する}%タイトル
\author{ER18028 久保 宏樹}%著者
\date{2020年12月14日}%日付

\begin{document}%本文の始まり
\maketitle%上記で設定したtitle,author,dateを表示
\section{画像の表示について}
chapter2\_image.texとchapter2\_image.pdfを照らし合わせて読んでください.\\

usepackageでdvipdfmx, graphicx, floatを指定すると画像を埋め込むことが可能です.\\

図は,図\ref{fig:label}のように参照することが可能です.基本的に,図を記述する場合,必ず1度以上参照する必要があります.
また,一度コンパイルしただけだと,参照した時に「?」と表示されるので「pLaTeX(ptex2pdf)」→「pLaTeX(ptex2pdf)」と2回コンパイルを実行する必要があります.\\
※ VScodeでの環境構築が完了している場合,「ctrl+s」で保存するだけでコンパイルが完了します.

%---------------図の表示方法--------------------
%------------------------------------------

%---------------以下課題-------------------------------
\section{課題}
\noindent 必須:authorを自分の名前,dateを提出日に変更してください.\\
任意:図を追加してください(図は何でも可).また,refを使用して「図を図\ref{fig:label}に示す」の様に参照してください.\\

\noindent 編集が完了したら,コンパイルしてください.\\
VScodeで編集している場合,「ctrl+s」で保存&コンパイル→「ctrl+alt+V」で出力されるpdfをプレビューできるので変更点が反映されていることを確認してください.
%--------------------------------------------------------
\end{document}

%以下,今回登場した機能の説明
%----------------------図----------------------------------------------
%\begin{figure}[画像の表示位置]                       % 表の表示位置の例→h:その場,b:ページ下部,t:ページ上部,p:他のページ
%    \begin{center}                                  % \begin{center}から\end{center}の間までが中心に表示される
%        \includegraphics[width=横幅]{画像のパス}       % 画像までのパスは相対パス
%    \end{center}
%    \caption{図のタイトル}                                                
%    \label{ラベル名}                   %図の場合は\label{fig:ラベル名}で定義することが多い.ラベルは\ref{ラベル名}で参照可能.
%\end{figure}
%------------------------
