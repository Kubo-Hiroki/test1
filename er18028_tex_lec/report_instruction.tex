%----以下文字の大きさ,フォント,余白などを決定する設定----
\documentclass[a4paper,10pt]{jsarticle}
\bibliographystyle{junsrt} %参考文献の並べ順junsrtの場合,本文内で参照した順
\setlength{\topmargin}{-17truemm}
\setlength{\oddsidemargin}{-0.4truemm}
\setlength{\textwidth}{160truemm}
\setlength{\textheight}{247truemm}
%-------------------------------------------------


\title{レポートについて}%タイトル
\author{ER17026 木南 貴志}%著者
\date{2020年12月14日}%日付

\begin{document}%本文の始まり
\maketitle
\section{レポートの内容}
下記の2つの内容についてまとめたレポートを作成しなさい.
\begin{itemize}
    \item ニューラルネットワークについて\\
    \item 「誤差逆伝播法」「ディープラーニング」「畳み込みニューラルネットワーク」の中のどれか一つについて
\end{itemize}

\section{注意点}
レポート中に下記の要素を含んでください.
\begin{itemize}
    \item セクションを使用すること
    \item 図を一つ以上挿入すること.その際,refを用いて文章中で各図番号を1度以上参照すること. 
    \item 表を一つ以上挿入すること.その際,refを用いて文章中で各表番号を1度以上参照すること.
    \item 式を一つ以上挿入すること.その際,refを用いて文章中で各式番号を1度以上参照すること.また,式の場合は式番号に括弧をつけること. 
    \item bibtexを用いて参考文献を一つ以上挿入すること.その際,citeを用いて文章中で各参考文献を1度以上参照すること.
    \item しおりの表示,ハイパーリンクが行えるようにすること(usepackageでhyperrefとpxjahyperを指定するだけでOK).
    \item コードの埋め込みはどちらでもOK.ただし挿入する場合は,refを用いて文章中で各コード番号を1度以上参照すること. 
    \item レポートのページ数には特に指定はありません.
\end{itemize}

\section{提出の際の注意点,提出期限}
レポート提出の際はtexファイルと,コンパイルした後のpdfの2つのファイルを提出してください.
また,ファイルの名前は「er18○○○report」にしてください.\\
レポートの提出期限は1月18日(日)の18時までとします.

\end{document}