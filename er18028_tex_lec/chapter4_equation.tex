%----以下文字の大きさ,フォント,余白などを決定する設定----
\documentclass[a4paper,10pt]{jsarticle}
\setlength{\topmargin}{-17truemm}
\setlength{\oddsidemargin}{-0.4truemm}
\setlength{\textwidth}{160truemm}
\setlength{\textheight}{247truemm}
%-------------------------------------------------
%\usepackage{}:Latexで拡張機能を使用するための宣言-------------
\usepackage{amsmath}%数式関連のパッケージ(alignなど)
\usepackage{bm}%太字でベクトルを表記
%----------------------------------------------------------

\title{式を表示する}%タイトル
\author{ER18028 久保 宏樹}%著者
\date{2020年12月20日}%日付

\begin{document}%本文の始まり
\maketitle%上記で設定したtitle,author,dateを表示
\section{式の表示について}
chapter4\_equation.texとchapter4\_equation.pdfを照らし合わせて読んでください.\\

usepackageでamsmathを指定すると式を記述可能になります.またbmを指定すると式内で太文字のベクトルが記述可能になります.\\

式は,式(\ref{eq:label1})(\ref{eq:label2})のように参照することが可能です.基本的に,式を記述する場合,必ず1度以上参照する必要があります.
また,一度コンパイルしただけだと,参照した時に「?」と表示されるので「pLaTeX(ptex2pdf)」→「pLaTeX(ptex2pdf)」と2回コンパイルを実行する必要があります.\\
※ VScodeでの環境構築が完了している場合,「ctrl+s」で保存するだけでコンパイルが完了します.
%---------------式の表示------------------
\begin{gather}
    x_{1} = L_{1} \cos \theta_{1} 
    \label{eq:label1} \\
    y_{1} = L{1}\sin \theta_{1} 
    \label{eq:label2}
\end{gather}
%--------------------------------------------

%---------------数式の表示--------------------
文章中に$x_1$のように,数式を表示させることも可能です.(ただし,文章中の数式には数式番号がつかない)
%------------------------------------------
%-----------------以下課題----------------------------
\section{課題}
\noindent 必須:authorを自分の名前,dateを提出日に変更してください.\\
任意:式を追加してください(式は何でも可).また,refを使用して「式を式(\ref{eq:label1})に示す」の様に参照してください.
式の場合は式\ref{eq:label1}ではなく,式(\ref{eq:label1})の様に,括弧で囲んでください.\\

\noindent 編集が完了したら,コンパイルしてください.\\
VScodeで編集している場合,「ctrl+s」で保存&コンパイル→「ctrl+alt+V」で出力されるpdfをプレビューできるので変更点が反映されていることを確認してください.
%------------------------------------------------------
\end{document}

%以下,今回登場した機能の説明
%-----------方程式------------------
%\begin{gather}
%   方程式1 
%   \label{ラベル名} \\ %数式の場合は\label{eq:ラベル名}で定義することが多い.ラベルは\ref{ラベル名}で参照可能.
%   方程式2
%   \label{ラベル名}  
%\end{gather}
%------------------------------------

%-------------数式--------------------
% $数式$
%-------------------------------------