\documentclass[a4paper,10pt]{jsarticle}
\setlength{\topmargin}{-17truemm}
\setlength{\oddsidemargin}{-0.4truemm}
\setlength{\textwidth}{160truemm}
\setlength{\textheight}{247truemm}
%-------------------------------------------------
\usepackage{listings,jlisting}%ソースコードを埋め込むためのパッケージ(jlistingはサイトからDLする必要あり)
%Windowsにjlistingパッケージを導入する https://mildtech.hatenablog.com/entry/2017/07/24/160324
%------------------------------------------------------------------

\lstset{%ソースコードを書き込む時の設定--------------------------------
  language=python,%言語
  columns=[l]{fullflexible}, % 文字をつめる
  basicstyle=\footnotesize,%
  commentstyle=\textit,%
  classoffset=1,%
  keywordstyle=\bfseries,%
	frame=tRBl,framesep=5pt,%枠線の種類 frame=singleならば一重枠
  showstringspaces=false,%
  numbers=left,stepnumber=1,numberstyle=\footnotesize,%
  breaklines=true,%trueで1行が長い時に改行
  firstnumber=1,%行番号が何番から始まるか
  xrightmargin=1zw,%枠右側の余白 
  xleftmargin=1zw%枠左側の余白
}%本文中に \lstinputlisting[caption=キャプション,label=ラベル]{ファイル名}でソースコードを貼り付け
%--------------------------------------------------------------------

\title{コードの埋め込み}%タイトル
\author{ER18028 久保 宏樹}%著者
\date{2020年12月20日}%日付

\begin{document}%本文の始まり
\maketitle%上記で設定したtitle,author,dateを表示
\section{コードの埋め込みについて}
chapter7\_code.texとchapter7\_code.texを照らし合わせて読んでください.\\

usepackageでlistingsを指定すると,ソースコードを埋め込むことができます.また,jlistingを使用すると日本語にも対応しますが,
デフォルトではインストールされていないので「semi/homework/2020\_12\_14/2020\_12\_14.ipynb」を参考にして導入してください.\\

ソースコードは,ソースコード\ref{code:label}のように参照することが可能です.基本的に,ソースコードを記述する場合,必ず1度以上参照する必要があります.
また,一度コンパイルしただけだと,参照した時に「?」と表示されるので「pLaTeX(ptex2pdf)」→「pLaTeX(ptex2pdf)」と2回コンパイルを実行する必要があります.\\
※ VScodeでの環境構築が完了している場合,「ctrl+s」で保存するだけでコンパイルが完了します.
\lstinputlisting[caption = title, label = code:label]{code/code.py}
%\lstinputlisting[caption = title, label = code:label, language=c++]{code/code.py} %左のようにすれば本文中で設定を変更することも可能です(左例ではプログラミング言語をpython→c++に変更しています).
%----------------以下課題-------------------------------------
\section{課題}
\noindent 必須:authorを自分の名前,dateを提出日に変更してください.\\
任意:コードを追加してください(コードは何でも可).また,refを使用して「コードをコード\ref{code:label}に示す」の様に参照してください.\\

\noindent 編集が完了したら,コンパイルしてください.\\
VScodeで編集している場合,「ctrl+s」で保存&コンパイル→「ctrl+alt+V」で出力されるpdfをプレビューできるので変更点が反映されていることを確認してください.\\
%-------------------------------------------------------------------
\end{document}

%以下,今回登場した機能の説明
%---------------コードの記述-------------------
%\lstinputlisting[設定]{コードまでのパス} %コードまでのパスは相対パスです.
%---------------------------------------------



