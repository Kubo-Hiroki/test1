%----以下文字の大きさ,フォント,余白などを決定する設定----
\documentclass[a4paper,10pt]{jsarticle}
\bibliographystyle{junsrt} %参考文献の並べ順を設定可能.junsrtの場合,本文内で参照した順
\setlength{\topmargin}{-17truemm}
\setlength{\oddsidemargin}{-0.4truemm}
\setlength{\textwidth}{160truemm}
\setlength{\textheight}{247truemm}
%-------------------------------------------------

%\usepackage{}:Latexで拡張機能を使用するための宣言-------------
\usepackage[dvipdfmx]{hyperref}%しおり作成,URL作成,ハイパーリンク機能のためのパッケージ(bibtexに直接は関係ありません)
%------------------------------------------------------------------

\title{bibtexを使用して参考文献を表示する}%タイトル
\author{ER18028 久保 宏樹}%著者
\date{2020年12月20日}%日付

\begin{document}%本文の始まり
\maketitle%上記で設定したtitle,author,dateを表示
\section{bibtexについて}
chapter5\_bibtex.texとchapter5\_bibtex.texとreference.bibを照らし合わせて読んでください.\\

bibファイルを作成して,そこに参考文献の情報を記述します.ここに記述するデータベースをbibtexと言います.今回は,「reference.bib」という名前にしました.\\
有名な論文な場合,「論文名 bibtex」と検索するとbibtexがヒットすることが多いです(他に調べる方法や,自分で作成する方法は気になったら調べてください).
今回はLSTMという論文の情報を「LSTM bibtex」で検索して調べました.bibtexの記述があるサイトが見つかると思うので,基本的にはそれをbibファイルにそのままコピー&ペーストすればOKです.\\

LSTM\cite{LSTM}のように,参考文献を参照することが可能です.一度も参照されていない文献はページ下部の「参考文献」の欄にも表示されないので注意してください.
また,bibtexは通常のコンパイルでは,反映されないため「pLaTeX(ptex2pdf)」→「pBibTeX」→「pLaTeX(ptex2pdf)」→「pLaTeX(ptex2pdf)」の順番でコンパイルする必要があります.\\
※ VScodeで環境構築が完了している場合,「LaTeX Workshop:Build with recipe」を実行するショートカットを入力→「build-and-bibtex」を選択してください.\\

chromeの拡張機能bibtex entry from url\cite{bibtex_url}を使用すれば,webサイトのbibtex情報も簡単に取得することが可能です.
ただし,urlを記述する場合はusepackageにhyperrefを指定する必要があります.
%-----------------以下課題--------------------------------------------
\section{課題}
\noindent 必須:authorを自分の名前,dateを提出日に変更してください.\\
任意:reference.bibに参考文献を追加してください(参考文献は何でも可).また,citeを使用して「LSTM\cite{LSTM}」の様に参照してください.\\

\noindent 編集が完了したら,コンパイルしてください.\\
VScodeで編集している場合,「LaTeX Workshop:Build with recipe」を実行するショートカットを入力→「build-and-bibtex」でコンパイル→「ctrl+alt+V」で出力されるpdfをプレビューできるので変更点が反映されていることを確認してください.
%--------------------------------------------------------------
\bibliography{reference.bib}%bibtexを読み込む
\end{document}

\end{document}
